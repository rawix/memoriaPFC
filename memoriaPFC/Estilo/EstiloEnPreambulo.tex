
% %%%%%%%%%%%%%%%%%%%%%%%%%%%%%%%%%%%%%%%%%%%%%%%%%%%%%%%%%%%%%%%%%%
% %%% DEFINICIONES DE ESTILO QUE DEBEN INCLUIRSE EN EL PRE�MBULO %%%
% %%%%%%%%%%%%%%%%%%%%%%%%%%%%%%%%%%%%%%%%%%%%%%%%%%%%%%%%%%%%%%%%%%


% FORMATO DE P�GINA (LAYOUT) ______________________________________________________________________

\setlength{\hoffset}{0pt}
% [1] M�s una pulgada (1 pulgada = 2.53807cm = 72pt), distancia desde la izquierda al margen de impresi�n.
\setlength{\voffset}{-15pt}
% [2] M�s una pulgada (1 pulgada = 2.53807cm = 72pt), distancia desde arriba al margen de impresi�n.
\setlength{\oddsidemargin}{10pt}
% [3] En las p�ginas IMPARES Distancia, por la izquierda, desde el margen de impresi�n, hasta CUERPO del texto.
\setlength{\evensidemargin}{28pt}
% [3] En las p�ginas PARES: Distancia, por la izquierda, desde el margen de impresi�n, hasta CUERPO del texto.
\setlength{\topmargin}{0pt}
% [4] Distancia, por arriba, desde el margen de impresi�n, hasta la CABECERA.
\setlength{\headheight}{15pt}
% [5] Alto de la CABECERA.
\setlength{\headsep}{17pt}
% [6] Distancia, por arriba, desde la CABECERA hasta el CUERPO del texto.
\setlength{\textheight}{666pt}
% [7] Alto del CUERPO del texto (23,48 cm).
\setlength{\textwidth}{416pt}
% [8] Ancho del CUERPO del texto (14,66 cm).
\setlength{\marginparsep}{7pt}
% [9] Distancia, por la derecha, desde el CUERPO del texto, hasta las NOTAS AL MARGEN.
\setlength{\marginparwidth}{54pt}
% [10] Ancho de las NOTAS AL MARGEN.
\setlength{\footskip}{30pt}
% [11] Distancia, por abajo, entre el CUERPO del texto y el PIE, m�s el alto del PIE.
%\setlength{\marginparpush}{5pt}
%



% FORMATO DE CABECERA Y PIE DE P�GINA _____________________________________________________________

\pagestyle{fancy}
\renewcommand{\chaptermark}[1]{\markboth{#1}{}}
\renewcommand{\sectionmark}[1]{\markright{#1}}
\fancyhf{}
% Vaciar cabecera y pie de p�gina.
\fancyhead[RO]{{\footnotesize{\scshape\leftmark\ /} \thepage}}
% En las IMPARES, a la derecha: T�tulo del cap�tulo, Barra, N�mero de p�gina.
\fancyhead[LE]{{\footnotesize \thepage {\scshape\ /\ \leftmark}}}
% En las PARES, a la izquierda: N�mero de p�gina, Barra, T�tulo del cap�tulo.
\renewcommand{\headrulewidth}{0pt}
% Eliminar la l�nea inferior.



% FORMATOS ESPECIALES EN T�TULOS DE CAP�TULOS, SECCIONES,... ______________________________________

\makeatletter

\renewcommand{\@makechapterhead}[1]{\vspace*{100\p@}%
{\parindent \z@ \raggedright \normalfont
     \ifnum \c@secnumdepth > \m@ne
       \if@mainmatter
         \large\scshape
         \textbf{\@chapapp}\space
         \Huge\scshape \textbf{\thechapter}
         \par\nobreak
         \vskip 0\p@
       \fi
     \fi
\interlinepenalty\@M \LARGE \scshape \textbf{#1}\par\nobreak
\vskip 0\p@
}%
\vspace*{25\p@}}

\renewcommand{\section}{\@startsection {section}{1}{\z@}%
                                   {-3ex \@plus -1ex \@minus -.2ex}%
                                   {1.25ex \@plus.2ex}%
                                   {\normalfont\large\bfseries}}

\renewcommand{\subsection}{\@startsection{subsection}{2}{\z@}%
                                     {-2.25ex\@plus -1ex \@minus -.2ex}%
                                     {1ex \@plus .2ex}%
                                     {\normalfont\normalsize\bfseries}}

\renewcommand{\subsubsection}{\@startsection{subsubsection}{3}{\z@}%
                                     {-2.25ex\@plus -1ex \@minus -.2ex}%
                                     {1ex \@plus .2ex}%
                                     {\normalfont\normalsize\bfseries}}
\makeatother



% FORMATOS ESPECIALES EN T�TULOS DE ENTORNOS ______________________________________________________

\captionsetup{margin=50pt,font=small,labelfont=bf,labelsep=period}
%,aboveskip=16pt
\newcommand{\Titulo}[2]{
    \caption[#1] {\textbf{#1.} #2.}}
% T�tulo habitual en figuras y tablas.

\newcommand{\TituloB}[1]{
    \caption[]{\textbf{#1.}}}
% T�tulo reducido en figuras y tablas (NO incluido en la lista de figuras o tablas)

\newcommand{\TituloC}[1]{
    \caption[#1]{\textbf{#1.}}}
% T�tulo reducido en figuras y tablas (S� incluido en la lista de figuras o tablas)


% FORMATOS GENERALES ______________________________________________________________________________

\linespread{1.1}
% Espacio entre l�neas (Interlineado).
\sloppy
% Para recortar convenientemente las citas bibliogr�ficas.
\deactivatetilden
% Para no confundir "~N" con la �.

%\setglosslabel{\rmfamily\bfseries#1\ifglossshort{ (#3)}{}}
% Formato de las entradas en el Glosario.

\renewcommand{\mathbf}[1]{\boldsymbol{#1}}
% Para imprimir negrita y cursiva en modo ecuaci�n.

%\setcounter{secnumdepth}{4} \setcounter{tocdepth}{4}
% Para cambiar la profundidad de numeraci�n en secciones e �ndice.




