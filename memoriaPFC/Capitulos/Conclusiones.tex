\chapter{Conclusiones finales}\label{CAP:Conclusiones}
Llegados a este cap\'itulo, s\'olo queda realizar una valoraci\'on de todo el trabajo realizado para dar cierre a esta memoria. Se realizar\'a un resumen resaltando los puntos m\'as importantes del proyecto, los objetivos adquiridos, una valoraci\'on personal del trabajo llevado a cabo y posibles l\'ineas futuras para poder continuar con el trabajo de esta idea tan divertida y �til.\\

\section{Etapas del proyecto}
Las etapas de este trabajo han sido muy diversas, teniendo cada una de ellas una dificultad diferente y una importancia en el resultado final. Como primera etapa consideramos la planificaci\'on y organizaci\'on de los meses que compondr\'ian dicho trabajo. Dicha etapa fue bastante importante ya que es cuando hay que organizar y definir los objetivos que se deben cumplir durante el desarrollo, y dentro de unos plazos determinados, que a lo largo del trabajo ir�n modific�ndose para adaptarse a los tiempos reales. Tras esta etapa de planificaci\'on llega la etapa de desarrollo, donde se comienza a construir poco a poco la idea de este proyecto, donde la idea comienza a tomar forma. Y para el cierre de este apartado de etapas llega la etapa de las pruebas y la valoraci�n de los usuarios.\\

\section{Conocimiento adquirido}
Como todo proyecto, cuando se comienza a planificar y a decidir qu\'e herramientas se usar\'an, qu\'e tecnolog\'ias se aplicar\'an en el desarrollo, el conocimiento de las mismas no es completo. Mucho m\'as del 50\% del conocimiento adquirido a lo largo de la carrera en este proyecto se ha afianzado a la vez que se ha ampliado.\\

Muchas de las tecnolog\'ias usadas, se han estudiado detenidamente para conocer qu\'e ventajas y utilidades podr\'ian ser provechosas para este proyecto.\\

\section{Objetivos conseguidos}
Al principio de la memoria, se realiz\'o un listado de objetivos que deb\'ian cumplirse cuando estuviese realizado todo el trabajo. Pero para llegar a cumplir el objetivo final, se han tenido que ir realizando una serie de hitos para cumplir peque\~nos objetivos, que enriquecen a nuestro proyecto de una gran organziaci\'on.\\

\subsection{Objetivo 1. Funcionalidad b\'asica de la aplicaci�n web.}
Este proyecto fue una idea planteada por uno de los profesores de la escuela de la ETSIT. La idea general de lo que se quer\'ia conseguir con esta aplicaci\'on web era una idea a la que faltaba darle forma y color. Pero antesd e comenzar a darle forma y color necesitabamos darle un significado, es decir, saber con claridad la funcionalidad, com\'o se quer\'ia que los usuarios interactuasen a trav\'es de ella, qu\'e se quer\'ia ofrecer a los usuarios y de qu\'e manera se pretende hacer uso de ella.\\

\subsection{Objetivo 2. Usuarios de la aplicaci�n web.}
A pesar de ser conscientes de que en internet hay una inmensa diversidad de webs y herramientas orientadas a los alumnos erasmus, el objetivo de dicha herramienta, es exclusivamente los alumnos y profesores de la escuela de la ETSIT. Por tanto, todo aquel que podr\'a usarla de ser alumno o prefoser de la URJC, siendo m\'as \'util la aplicaci\'on web que si eprteneciesen a otra universidad cualquiera de la geograf\'ia nacional.\\

Decididos los usuarios a los que va dirigida dicha aplicaci\'on, iremos con m\'as firmeza a la hora de tomar las decisiones de la implementaci\'on del c\'odigo.\\ 

\subsection{Objetivo 3. Bases de datos.}
Una vez definida la funcionalidad de la herramienta, y el tipo de usuario a la que va dirigida se procedi\'o a definir el tipo de informaci\'on y datos que era preciso almacenar para cumplir con la funcionalidad deseade de la aplicaci\'on. Este punto es bastante importante, ya que si no se definen correctamente los datos a utilizar puede dar lugar a fallos en la funcionalidad.\\

Lo que se pretende con este objetivo, es tener claros los conceptos que obtendr\'an los usuarios al utilizarla, y para ello la definici\'on de las tablas de la base de datos a usar fue algo decisivo. Ya que a estos datos se accede desde cualquier punto de la aplicaci\'on web, por lo que si no tenemos bien estructurados dichos datos nos resultar\'ia m\'as trabajoso ir avanzando con los hitos siguientes.\\

\subsection{Objetivo 4. Aplicaci�n web.}
Ya definidas las tablas de la base de datos, el siguiente objetivo a cumplir es el desarrollo e implementaci\'on de la aplicaci\'on web. Como primera para decidir c�mo acceder a los distintos tipos de datos, y c\'omo mostrarlos al usuario para luego poder definir el dise\~no de la interfaz de usuario, tanto para las p\'aginas p\'ublicas (usuarios no registrados) como para las privadas (usuarios registrados), para que sea intuitiva al uso. Por supuesto ha sido la etapa que m\'as tiempo ha llevado, ya que compone los meses de trabajo m\'as complejo, la construcci\'on de todo lo que conforma la aplicaci\'on.\\

\subsection{Objetivo 5. Hosting web}
Una vez implementada la aplicaci\'on web y el dise\~no de la interfaz de usuario, la siguiente decisi\'on que se tom\'o fue c\'omo realizar el despliegue, para que los alumos pudiesen probar la herramienta, y conocer de primera mano qu\'e puntos hay que mejorar del trabajo. En primer lugar se pens\'o en usar una m\'aquina virtual en el campus de la universidad. Pero se vi\'o que era m\'as \'optimo realizar el despliegue usando un PC como servidor de la aplicaci\'on. Aunque el \'ultimo despliegue realizado se realiz\'o mediante un \textit{hosting} web.\\

\section{L�neas futuras}
Una vez finalizado el proyecto, y habiendolo usado con un mayor n\'umero de usuarios, se pudo observar que era necesario incluir algunas caracter\'isticas que pudiesen motivar e incentivar el uso de la aplicaci\'on. En los siguientes p\'arrafos vamos a comentar en profundidad las mejoras que se proponen.\\

\subsubsection{Gamificaci\'on}
Gamificaci\'on es el empleo de mec\'anicas de juego en entornos y aplicaciones no l\'udicas con el fin de potenciar la motivaci\'on, la concentraci\'on, el esfuerzo, la fidelizaci\'on y otros valores positivos comunes a todos los juegos. Se trata de una nueva y poderosa estrategia para influir y motivar a grupos de personas.\\

La eclosi\'on de la web 2.0 ha acelerado la creaci\'on de comunidades en torno a todo tipo de redes sociales, medios digitales o webs corporativas. Pero no siempre es f\'acil estimular la actividad din\'amica y frecuente entre los miembros de una comunidad. Una correcta implementaci\'on de estrategias de gamificaci\'on permite pasar de la mera conectividad al \textit{engagement} (o compromiso), logrando que los miembros de una comunidad, los trabajadores de una empresa, los estudiantes de un instituto, los habitantes de una ciudad participen de manera din\'amica y proactiva en acciones que generalmente requieren un esfuerzo de la voluntad.\\

La integraci\'on de din\'amicas de juego en entornos no l\'udicos no es un fen\'omeno nuevo, pero el crecimiento exponencial del uso de videojuegos en los \'ultimos a\~nos ha despertado el inter\'es de expertos en comunicaci\'on, psicolog\'ia, educaci\'on, salud, productividad pos descifrar las que hacen del videojuego un medio tan eficaz.\\

En estos \'ultimos a\~nos ha comenzado tambi\'en la expansi\'on en el estudio de su aplicaci\'on a otros \'ambitos no necesariamente l\'udicos. Gamificaci\'on es el t\'ermino escogido para definir esta tendencia.\\

Aplicar gamificaci\'on a este proyecto ser\'ia una manera importante e ingeniosa de conseguir que los usuarios disfrutasen satisfactoriamente con el uso de la misma, asi como conseguir que hubiese una mayor n\'umero de usuarios y por tanto, un mayor uso de la misma.\\

\subsubsection{Interfaz m\'ovil}
Las aplicaciones -tambi\'en llamadas apps- llevan presentes en los tel\'efonos desde hace tiempo, de hecho, ya estaban incluidas en los sistemas operativos de Nokia o Blackberry a\~nos atr\'as. Los m\'oviles de esa \'epoca, contaban con pantallas reducidas y muchas veces no t\'actiles, y son los que ahora llamamos \textit{feature phones}, en contraposici\'on a los \textit{smartphones}, m\'as actuales.\\

\begin{figure}[htbp]
	
	\centering
	\includegraphics[scale=0.2]{./Figuras/diversasapps.jpg}
	\caption{En la AppStore hay cientos de miles de apps disponibles. }
	\label{fig:html5}    
	
\end{figure}

Actualmente encontramos aplicaciones de todo tipo, forma y color, pero en los primeros tel\'efonos, estaban enfocados en mejorar la productividad personal: se trataba de alarmas, calendarios, calculadoras y clientes de correo.\\

Hubo un cambio grande con el ingreso del iPhone al mercado, ya que con \'el se generaron nuevos modelos de negocio que hicieron de las aplicaciones algo rentable, tanto para desarrolladores como para los mercados de aplicaciones, como \textit{App Store}, \textit{Google Play} y \textit{Windows Phone Store}.\\

Al mismo tiempo, tambi\'en mejoraron las herramientas de las que dispon\'ian dise\~nadores y programadores para desarrollar apps, facilitando la tarea de producir una aplicaci\'on y lanzarla al mercado, incluso por cuenta propia. De aqu\'i la idea de crear una interfaz m\'ovil para esta aplicaci\'on web, a\'un sabiendo que las tecnolog\'ias usadas para la interfaz de usuario son tecnolog\'ias din\'amicas, por lo que se adaptan tanto a los navegadores de las tablets, de los smartphones y de los PCs.\\

\subsubsection{Diferencias entre aplicaciones y web m\'oviles}
Las aplicaciones comparten la pantalla del tel\'efono con las webs m\'oviles, pero mientras las primeras tienen que ser descargadas e instaladas antes de usar, a una web puede accederse simplemente usando Internet y un navegador; sin embargo, no todas pueden verse correctamente desde una pantalla generalmente m\'as peque\~na que la de un ordenador de escritorio.\\

Las que se adaptan especialmente a un dispositivo m\'ovil se llaman "web responsivas" y son ejemplos del dise\~no l\'iquido, ya que se puede pensar en ellas como un contenido que toma la forma del contenedor, mostrando la informaci\'on seg\'un sea necesario. As\'i, columnas enteras, bloques de texto y gr\'aficos de una web, pueden acomodarse en el espacio de una manera diferente -o incluso desaparecer- de acuerdo a si se entra desde un tel\'efono, una tableta o un ordenador.\\

\section{Valoraci�n personal}
Desarrollar aplicaciones web no se parece en nada a lo que en un principio cre\'ia. Hay que tener las ideas bastante claras, y saber qu\'e usar en cada uno de los pasos que se van dando. Esta experiencia me ha ense\~nado a saber c\'omo organizar un proyecto tan largo y complejo como este.\\

Como conclusi\'on de este proyecto, que destaca por el duro trabajo y el amplio tiempo dedicado, es que no importan los momentos cr\'iticos, ya que con paciencia y tranquilidad todo termina solucionandose, y el poder observar lo que se ha creado supera con creces cualquiera duro momento.\\

Parece que fue ayer cuando acud� al despacho de mi tutor, Gregorio, para hablar sobre el Proyecto Fin de Carrera. Tras estos meses de trabajo, ha llegado el momento de evaluar el tiempo dedicado al proyecto.\\

Desde el principio tuve bastante claro que quer\'ia como tutor del proyecto a Gregorio, el que ha sido mi tutor en este trabajo final. Como tutor ha sido paciente conmigo, ya que comenc\'e el proyecto estando en Granada, con lo que el \'unico medio de comunicaci\'on entre ambos fueron los correos electr\'onicos. Hasta que pude subir a Madrid, no tuvimos una reuni\'on para revisar la planificaci\'on ni el desarrollo de los primeros hitos. Adem\'as de esa paciencia, me ha sabido guiar durante el transcurso del proyecto, aconsejado en los momentos cr\'iticos, ense�ado a aprender a resolver los problemas que han ido surgiendo en el desarrollo del mismo. En los momento de bloqueo ha sabido darme las pautas necesarias para avanzar poco a poco y obtener lo que es hoy "TuErasmus".\\

Los meses de trabajo del proyecto, he tenido que compaginarlos con las labores como estudiante y al mismo tiempo con la beca que estoy realizando en Telef\'onica I+D. Ha sido un a\~no bastante duro, ya que he tenido que recortar todo aquello relacionado con mi vida social, ya que al tener disponibles s\'olo las horas del fin de semana, deb\'ia aprovechar al m\'aximo \'estas para poder avanzar en el proyecto.\\

Durante el desarrollo del proyecto me he enfrentado a situaciones en las que nunca me hab\'a encontrado. La gran mayor\'ia han sido problemas, los cuales tuve que solventar sola, es decir, intentar definir yo misma las pautas necesarias para llegar a la mejor soluci\'on. Claro est\'a que en ocasiones acertaba con la soluci\'on que escog\'ia y otras muchas volv\'ia a cometer el mismo error. Pero estas son las cosas que nos hacen madurar en cuanto a la experiencia, y poner nuestro conocimiento completamente en pr\'actica. Tambi\'en he tenido que decidir sobre temas de dise\~no o sobre la elecci\'on de herramietnas para su desarrollo derivadas de intensas labores de investigaci\'on, para tratar de encontrar siempre la mejor elecci\'on.\\

Por tanto el conocimiento adquirido tiene un valor incalculable, puesto que no s\'olo hay que contabilizar el volumen de conocimientos t\'ecnicos aprendidos, sino tambi\'en la experiencia adquirida, la necesidad de aprender a tomar decisiones, a resolver cuestiones y plantear soluciones a situaciones reales.\\

El desarrollo de este proyecto, tambi\'en me ha servido para poner en pr\'actica numerosos conocimientos que he ido adquiriendo a lo largo de esta carrera, sintiendo una gran satisfacci\'on al darme cuenta de que lo aprendido en la carrera es de gran utilidad. Una de las situaciones ante la cual estaba expectante, era saber que la aplicaci\'on iba a estar sometida a una prueba real. Sin duda alguna ha sido una de las mayores satisfacciones vividas, el poder probar con gente real algo que he creado yo misma. Es cierto, que al principio sientes miedo porque no sabes la reacci\'on que tendr\'a la gente ante lo que hab\'ia supuesto mi trabajo durante estos meses. Pero este miedo desapareci\'o en cuanto vi la aceptaci\'on positiva de algo que hab\'ia sido creado por m\'i, salvo po los peque\~nos errore que hubo que mitigar, esta ha sido, con diferencia, la mejor fase del proyecto.\\

Tras esta larga etapa, y haber pasado buenos y malos momentos, echo la vista atr\'as y estoy m\'as segura que nunca de que no cambiar\'ia nada de esta etapa tan bella y emotiva de la carrera. Espero que el lector, al igual que la autora de estas l\'ineas, haya disfrutado tanto de la lectura de estas palabras como del mensaje que se ha pretendido mandar a lo largo de todo el documento: \textit{Non nova, sed nove}.\\
